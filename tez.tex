\documentclass[msc]{kou}
% Bazı temel bilgiler.
\makeatletter
% Kaynak dosyası.
\addbibresource{kaynaklar.bib}
\addbibresource{kisisel.bib}
% Başlık
\baslik{Gemi Balast Tanklarının Otonom Denetimi İçin Reddedilmiş Ortamlara Yönelik Bütünleşik Robotik Sistem Mimarisi}
% İngilizce başlık
\thetitle{Integrated Robotic System Architecture for Autonomous Inspection of Ship Ballast Tanks in Denied Environments}
% Yazar (Soyad Büyük yazılması gerekmektedir.)
\yazar{Berker YİĞİT}
% Anabilim dalı - departman
\bolum{Bilgisayar Mühendisliği}
% Tarih
\tarihAY{Ocak}
\tarihYIL{2025}
% Savunma Tarihi
\starih{.../.../2025}
% Juri Üyeleri Adıları ile Açıklamaları.
% Danışman Bilgisi (Kullanıcı tarafından sağlandı)
\jrBIR{Dr. Öğr. Üyesi Alpaslan Burak İNNER}						
\jrBIRd{Danışman, Kocaeli Üniv.}
% Diğer Jüri Üyeleri (Taslak)
\jrIKI{Prof. Dr. Adı SOYADI}						
\jrIKId{Jüri Üyesi, Kocaeli Üniv.}
\jrUC{Dr. Öğr. Üyesi Adı SOYADI}						
\jrUCd{Jüri Üyesi, ... Üniv.}
% Yüksek Lisans tezlerinde genellikle 3 jüri yeterlidir ancak şablon gereği bırakılmıştır.
\jrDORT{Prof. Dr. Adı SOYADI}						
\jrDORTd{Jüri Üyesi, ... Üniv.}
\jrBES{Prof. Dr. Adı SOYADI}						
\jrBESd{Jüri Üyesi, ... Üniv.}
\makeatother
\begin{document}
\kapakolustur
\onayolustur
% Romen rakamlarının başlatılması
\clearpage
\pagenumbering{roman}
% Dizinleri  yazdırırken satır aralıklarının düzenlenmesi için
% grup içerisine alınmış ve düzenleme yapılmıştır.
\begingroup
\singlespacing
\etikolustur
\fikriolustur
% Önsöz dosyasının içeriği önceki adımlarda oluşturuldu.
\onsoz{Bu tez çalışması, denizcilik endüstrisinde insan hayatı için yüksek risk teşkil eden kapalı alan denetim süreçlerinin, otonom robotik sistemler aracılığıyla güvenli ve verimli bir şekilde gerçekleştirilebilirliğini araştırmaktadır. Kocaeli Üniversitesi Bilgisayar Mühendisliği Bölümü bünyesinde yürütülen bu çalışma, GPS ve manyetometre gibi temel navigasyon bileşenlerinin kullanılamadığı ''reddedilmiş ortamlarda'' görev yapabilen bütünleşik bir İnsansız Hava Aracı (İHA) mimarisini ve otonom navigasyon algoritmalarını ortaya koymaktadır.
Çalışmalarım boyunca akademik vizyonuyla bana yol gösteren, karşılaştığım teknik zorlukların aşılmasında bilgi ve tecrübelerini esirgemeyen değerli danışman hocama en içten teşekkürlerimi sunarım. Ayrıca, bu yoğun araştırma ve geliştirme sürecinde, özellikle donanım entegrasyonu ve saha testleri aşamalarında sabır ve destekleriyle her zaman yanımda olan aileme ve çalışma arkadaşlarıma minnettarım. Bu çalışmanın, denizcilik güvenliği ve otonom robotik sistemler alanında yapılacak gelecek araştırmalara katkı sağlamasını temenni ederim.
}

\onsozolustur
\renewcommand{\listfigurename}{Şekiller Dizini}
\renewcommand{\listtablename}{Tablolar Dizini}
\tableofcontents
\listoffigures
\listoftables
\chapter*{Simgeler ve Kısaltmalar Dizini}
\input{bolumler/simgeler.tex}
\tocless\section{Kısaltmalar}
\addtocounter{section}{1}
\input{bolumler/kisaltmalar.tex}
% Özet dosyasının içeriği (ozet.tex) önceki adımlarda oluşturuldu.
\ozet{
Bu tez çalışması, denizcilik endüstrisinde insan hayatı için yüksek risk oluşturan gemi su balast tanklarının denetim süreçlerinin, otonom insansız hava araçları (İHA) kullanılarak güvenli ve verimli bir şekilde gerçekleştirilmesini konu almaktadır. Çalışma kapsamında, GPS sinyallerinin bulunmadığı, manyetik pusulanın işlevsiz kaldığı ve fiziksel sınırların son derece kısıtlayıcı olduğu ''reddedilmiş ortam'' (denied environment) koşullarında görev yapabilen bütünleşik bir robotik sistem mimarisi tasarlanmıştır.
Geliştirilen sistem, 600x400 mm boyutlarındaki standart menhollerden geçişe uygun, çarpışmaya dayanıklı küresel bir kafes yapısına ve ferromanyetik pas tozuna dirençli bir itki sistemine sahiptir. Algılama katmanında, zenit ve nadir kör noktalarını ortadan kaldıran gimbal destekli ''nodding'' LiDAR teknolojisi ve stereo görsel sensörler kullanılarak çok katmanlı bir sensör füzyonu mimarisi oluşturulmuştur. Navigasyon ve haritalama süreçlerinde, ham nokta bulutu verilerini işleyen FAST-LIO2 algoritması ile gradyan tabanlı reaktif yörünge planlama sağlayan EGO-Planner algoritması, NVIDIA Jetson Orin NX gömülü sistemi üzerinde entegre edilmiştir.
Elde edilen bulgular, önerilen sistemin kapalı hacim aerodinamiği ve elektromanyetik izolasyon zorluklarına rağmen stabil uçuş gerçekleştirebildiğini ve Uluslararası Klas Kuruluşları Birliği (IACS) standartlarına uygun görsel veriler toplayabildiğini göstermektedir. Bu çalışma, tehlikeli kapalı alanlarda insan müdahalesini ortadan kaldıran endüstriyel bir robotik çözüm sunmaktadır.
}
\anahtarkelimeler{Otonom İHA, Gemi Balast Tankı, Reddedilmiş Ortam, Nodding LiDAR, FAST-LIO2, EGO-Planner, Sensör Füzyonu.}

\ozetolustur
% İngilizce Abstract dosyasının içeriği (abstract.tex) önceki adımlarda oluşturuldu.
\theabstract{Bu çalışma, GPS sinyallerinin bulunmadığı, yoğun manyetik parazitlerin yaşandığı ve fiziksel sınırların son derece dar olduğu gemi su balast tanklarının denetimi için geliştirilen bütünleşik bir otonom İnsansız Hava Aracı (İHA) sisteminin mimarisini ve mühendislik detaylarını sunmaktadır. Araştırma, menhollerden geçişe uygun küresel kafes tasarımı ve toza dayanıklı itki sistemlerini içeren mekanik yapıyı; gimbal destekli "nodding" LiDAR ve stereo vizyon teknolojilerini kullanan çok katmanlı bir sensör füzyonu ile entegre etmektedir. "Reddedilmiş ortam" (denied environment) koşullarında karşılaşılan geometrik yozlaşma, hareket bozulması (motion distortion) ve aerodinamik duvar etkilerini aşmak amacıyla, FAST-LIO2 tabanlı Lidar-Atalet odometrisi ve EGO-Planner tabanlı reaktif yörünge planlama algoritmalarının performansı analiz edilmiştir. Çalışma, NVIDIA Jetson Orin tabanlı gömülü sistem üzerinde kurgulanan ROS 2 yazılım mimarisini ve donanım bileşenlerini detaylandırarak, IACS standartlarına uygun, insan müdahalesi gerektirmeyen güvenilir bir denetim sistemi için gerekli teknik yol haritasını ortaya koymaktadır.}
\keywords{Otonom İHA, Balast Tankı Denetimi, Nodding LiDAR, FAST-LIO2, Sensör Füzyonu, ROS 2, EGO-Planner, GPS Reddedilmiş Ortam, SLAM, Robotik Sistem Mimarisi.}

\createabstract
\endgroup
% Latin/Arabic rakamların kullanılması
\clearpage
\pagenumbering{arabic}
% Bölümlerin Eklenmesi
% Not: 'tutorial.tex' yerine tezinizin ana bölümlerini (Giriş, Literatür, Materyal Yöntem vb.)
% içeren dosyaları buraya eklemelisiniz.
\input{bolumler/bolum1_giris.tex}
\input{bolumler/bolum2_literatur.tex}
\input{bolumler/bolum3_materyal_yontem.tex}
\input{bolumler/bolum4_bulgular.tex}
\input{bolumler/bolum5_sonuc.tex}
% Satır aralığının tekrar 1 satır olduğu Ekler,
% Kaynaklar, Kişisel yayın ve eser ile özgeçmiş bölümünün
% gruplandırılması
\begingroup
\singlespacing
\sloppy
\printbibliography[title={Kaynaklar}, notcategory=nobibliography]
% Ekler bölümü: Donanım listesi ve algoritmik detaylar (ekA.tex)
% önceki adımlarda oluşturulan içeriği kullanacaktır.
\eklerolustur
\section{Bütünleşik Otonom Sistem Donanım ve Yazılım Envanteri}
\label{sec:donanim_envanteri}

Bu ekte, gemi balast tanklarında gerçekleştirilecek otonom denetim görevleri için tasarlanan insansız hava aracının (İHA) teknik bileşenleri, seçim kriterleri ve operasyonel gerekçeleri detaylandırılmıştır. [span_0](start_span)[span_1](start_span)Sistem mimarisi; mekanik dayanıklılık, elektromanyetik izolasyon ve hesaplama verimliliği prensiplerine göre oluşturulmuştur[span_0](end_span)[span_1](end_span).

\subsection{Donanım Bileşen Listesi (Bill of Materials)}

Aşağıdaki Tablo \ref{tab:bilesen_listesi}, ''reddedilmiş ortam'' koşullarında operasyonel sürekliliği sağlamak üzere seçilen kritik donanım bileşenlerini özetlemektedir.

\begin{table}[htbp]
    \centering
    [span_2](start_span)\caption{Otonom Denetim Dronu İçin Seçilen Donanım Bileşenleri ve Teknik Gerekçeler[span_2](end_span).}
    \label{tab:bilesen_listesi}
    \renewcommand{\arraystretch}{1.3}
    \footnotesize
    \begin{tabular}{|p{0.25\textwidth}|p{0.35\textwidth}|p{0.3\textwidth}|}
    \hline
    \textbf{Alt Sistem} & \textbf{Bileşen / Özellik} & \textbf{Teknik Gerekçe} \\
    \hline
    [span_3](start_span)[span_4](start_span)\textbf{Gövde Mimarisi} & Karbon Fiber Küresel Kafes (Dış Çap $<$ 380mm) & Standart 600x400mm menhollerden geçiş ve Duvar Etkisi (Wall Effect) toleransı için gereklidir[span_3](end_span)[span_4](end_span). \\
    \hline
    [span_5](start_span)[span_6](start_span)\textbf{İtki Sistemi (Motor)} & 2203.5 veya 2004 Boyut, IP5X/IP6X, N52H Mıknatıs & Ferromanyetik pas tozuna karşı koruma ve yüksek ısı dayanımı sağlar[span_5](end_span)[span_6](end_span). \\
    \hline
    [span_7](start_span)\textbf{Pervaneler} & 3 - 3.5 inç, 3 veya 4 Bıçaklı Polikarbonat & Darbe anında kırılma yerine esneme yeteneği ve düşük gürültü profili sunar[span_7](end_span). \\
    \hline
    [span_8](start_span)[span_9](start_span)\textbf{Uçuş Kontrolcüsü} & Cube Orange+ veya Pixhawk 6X (Üçlü Yedekli IMU) & Manyetik parazitlere karşı dayanıklılık ve ısıtmalı IMU ile sensör kaymasını (drift) önler[span_8](end_span)[span_9](end_span). \\
    \hline
    [span_10](start_span)\textbf{Görev Bilgisayarı} & NVIDIA Jetson Orin NX (16GB RAM) & SLAM (FAST-LIO2) ve EGO-Planner algoritmaları için gerekli GPU gücünü sağlar[span_10](end_span). \\
    \hline
    [span_11](start_span)[span_12](start_span)\textbf{Lidar Sensörü} & Livox Mid-360 veya Ouster OS0 (Katı Hal) & 360 derece FOV ile tavan/zemin kör noktalarını ortadan kaldırır ve mekanik arıza riskini düşürür[span_11](end_span)[span_12](end_span). \\
    \hline
    [span_13](start_span)[span_14](start_span)\textbf{Görsel Sensörler} & Intel RealSense D455 (Navigasyon) + Sony Sensörlü 4K Kamera (Sörvey) & VIO (Görsel Odometri) için Global Shutter ve IACS standartlarında 0.8mm çatlak tespiti sağlar[span_13](end_span)[span_14](end_span). \\
    \hline
    [span_15](start_span)[span_16](start_span)\textbf{Aydınlatma} & 10.000+ Lümen, CRI $>$ 90, Eğik Montaj & Tozlu ortamda geri yansımayı (backscatter) önler ve korozyon detaylarını belirginleştirir[span_15](end_span)[span_16](end_span). \\
    \hline
    [span_17](start_span)\textbf{Güç Sistemi} & 6S LiPo Batarya, 75C+ Deşarj & Türbülanslı ortamda ani motor tork taleplerini karşılamak için yüksek voltaj ve akım stabilitesi sunar[span_17](end_span). \\
    \hline
    [span_18](start_span)[span_19](start_span)\textbf{İletişim} & COFDM Video Link + Sinyal Tekrarlayıcı (Repeater) & Çelik tank içindeki çok yollu sönümlenme (multipath fading) etkisine karşı direnç sağlar[span_18](end_span)[span_19](end_span). \\
    \hline
    \end{tabular}
\end{table}

\subsection{Yazılım ve Algoritma Yığını}

Donanım bileşenlerini işlevsel hale getiren yazılım mimarisi, GPS ve manyetometre verisi olmadan tam otonomi sağlayacak şekilde yapılandırılmıştır:

\begin{itemize}
    [span_20](start_span)\item \textbf{İşletim Sistemi:} Ubuntu 20.04 üzerinde ROS 2 (Humble/Jazzy) mimarisi[span_20](end_span).
    \item \textbf{Konumlandırma (SLAM):} FAST-LIO2 algoritması kullanılarak, LiDAR nokta bulutları ve IMU verileri sıkı sıkıya bağlı (tightly-coupled) bir filtrede birleştirilir. [span_21](start_span)[span_22](start_span)Bu yöntem, ani hareketlerde dahi yüksek doğruluk sağlar[span_21](end_span)[span_22](end_span).
    [span_23](start_span)[span_24](start_span)\item \textbf{Rota Planlama:} EGO-Planner algoritması ile ESDF haritası oluşturmadan, gradyan tabanlı optimizasyon ile milisaniye mertebesinde çarpışmasız yörüngeler hesaplanır[span_23](end_span)[span_24](end_span).
    [span_25](start_span)[span_26](start_span)\item \textbf{Kontrolcü:} ArduPilot (Copter 4.5+) yazılımı, manyetometre verisi devre dışı bırakılarak (EKF3\_MAG\_CAL=5:Never) ve yön bilgisi SLAM çıktısından alınarak yapılandırılmıştır[span_25](end_span)[span_26](end_span).
\end{itemize}

% Kişisel yayın ve eser bölümü
\begin{refsection}
\input{bolumler/kisiseleser.tex}
\end{refsection}
% Özgeçmiş bölümü
\input{bolumler/ozgecmis.tex}
\endgroup
\end{document}
