\ozet{
Bu tez çalışması, denizcilik endüstrisinde insan hayatı için yüksek risk oluşturan gemi su balast tanklarının denetim süreçlerinin, otonom insansız hava araçları (İHA) kullanılarak güvenli ve verimli bir şekilde gerçekleştirilmesini konu almaktadır. Çalışma kapsamında, GPS sinyallerinin bulunmadığı, manyetik pusulanın işlevsiz kaldığı ve fiziksel sınırların son derece kısıtlayıcı olduğu ''reddedilmiş ortam'' (denied environment) koşullarında görev yapabilen bütünleşik bir robotik sistem mimarisi tasarlanmıştır.
Geliştirilen sistem, 600x400 mm boyutlarındaki standart menhollerden geçişe uygun, çarpışmaya dayanıklı küresel bir kafes yapısına ve ferromanyetik pas tozuna dirençli bir itki sistemine sahiptir. Algılama katmanında, zenit ve nadir kör noktalarını ortadan kaldıran gimbal destekli ''nodding'' LiDAR teknolojisi ve stereo görsel sensörler kullanılarak çok katmanlı bir sensör füzyonu mimarisi oluşturulmuştur. Navigasyon ve haritalama süreçlerinde, ham nokta bulutu verilerini işleyen FAST-LIO2 algoritması ile gradyan tabanlı reaktif yörünge planlama sağlayan EGO-Planner algoritması, NVIDIA Jetson Orin NX gömülü sistemi üzerinde entegre edilmiştir.
Elde edilen bulgular, önerilen sistemin kapalı hacim aerodinamiği ve elektromanyetik izolasyon zorluklarına rağmen stabil uçuş gerçekleştirebildiğini ve Uluslararası Klas Kuruluşları Birliği (IACS) standartlarına uygun görsel veriler toplayabildiğini göstermektedir. Bu çalışma, tehlikeli kapalı alanlarda insan müdahalesini ortadan kaldıran endüstriyel bir robotik çözüm sunmaktadır.
}
\anahtarkelimeler{Otonom İHA, Gemi Balast Tankı, Reddedilmiş Ortam, Nodding LiDAR, FAST-LIO2, EGO-Planner, Sensör Füzyonu.}
