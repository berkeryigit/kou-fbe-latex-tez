\theabstract{Bu çalışma, GPS sinyallerinin bulunmadığı, yoğun manyetik parazitlerin yaşandığı ve fiziksel sınırların son derece dar olduğu gemi su balast tanklarının denetimi için geliştirilen bütünleşik bir otonom İnsansız Hava Aracı (İHA) sisteminin mimarisini ve mühendislik detaylarını sunmaktadır. Araştırma, menhollerden geçişe uygun küresel kafes tasarımı ve toza dayanıklı itki sistemlerini içeren mekanik yapıyı; gimbal destekli "nodding" LiDAR ve stereo vizyon teknolojilerini kullanan çok katmanlı bir sensör füzyonu ile entegre etmektedir. "Reddedilmiş ortam" (denied environment) koşullarında karşılaşılan geometrik yozlaşma, hareket bozulması (motion distortion) ve aerodinamik duvar etkilerini aşmak amacıyla, FAST-LIO2 tabanlı Lidar-Atalet odometrisi ve EGO-Planner tabanlı reaktif yörünge planlama algoritmalarının performansı analiz edilmiştir. Çalışma, NVIDIA Jetson Orin tabanlı gömülü sistem üzerinde kurgulanan ROS 2 yazılım mimarisini ve donanım bileşenlerini detaylandırarak, IACS standartlarına uygun, insan müdahalesi gerektirmeyen güvenilir bir denetim sistemi için gerekli teknik yol haritasını ortaya koymaktadır.}
\keywords{Otonom İHA, Balast Tankı Denetimi, Nodding LiDAR, FAST-LIO2, Sensör Füzyonu, ROS 2, EGO-Planner, GPS Reddedilmiş Ortam, SLAM, Robotik Sistem Mimarisi.}
